\section*{Abstract}
This thesis investigates the potential of distributed robotics systems in automating the assembly of timber structures, addressing the challenges of large-scale spatial manipulation and tight-fitting timber joint assembly, which are highly relevant for timber construction.

Leveraging the highly automated process of machining timber parts using automatic joinery machines, the thesis investigates the next knowledge gap in the design-to-production workflow - automatic spatial assembly.
 Using timber frame structures with integral timber joints as a starting point, this thesis proposed a new fabrication system using distributed robotic tools (DiRT) in collaboration with industrial robotic arms.
 The crucial breakthrough is the modular and remote operation nature of the tools, allowing the system to assemble a wide variety of timber joints and complex structures.

This thesis also investigated an integrated design workflow.
 Design validation is identified as a critical aspect of the automated assembly process.
 This research proposes a practical three-tier validation process to evaluate a design, with quick feasibility feedback provided to the designer during the design process.
 It takes into consideration geometrical conflicts, robot limitations and tool setup to provide visual feedback on various problems to the designer.
 
The research provides a proof-of-concept through the development of three full-scale timber frame demonstrators, each assembled using a single robotic arm and a set of custom-designed distributed assembly tools.
 The assembly tools include robotic clamps and screwdrivers for different types of lap joints, including planar and non-planar varieties.
 The findings showcased a viable method to assemble timber structures, mitigating well-known problems such as accumulated assembly error and instability during construction.
 The results also identified key challenges that are limiting the system efficiency, accuracy, reliability and success rate for the automated process, as well as discovering new opportunities for future research.
 These opportunities include establishing a generalizable DiRT assembly system, and expanding the range of joint types and building components that can be assembled.

The thesis contributes software tools and system design patterns that are generalizable and reusable within the broader digital fabrication and construction automation community.
 For example, software for remote robot operation and synchronisation; Data structures and algorithms for robotically assembled structures; Methods for automating parsing designs into robotic programmes; and Task and motion planning techniques for assembly problems.

Ultimately, this research contributes to ongoing efforts to harness the potential of robotics for creating more efficient and sustainable timber construction processes.
 Paving the way for the widespread adoption of automated construction processes within the architectural industry.

