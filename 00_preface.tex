\begin{adjustwidth}{20mm}{20mm}
\chapter*{Preface}
\addcontentsline{toc}{chapter}{Preface}

In the rapidly changing world of architecture and construction, we stand on the brink of a new era where robotics, industrial automation and computational design practices will substantially change the method of construction. As we confront the challenges of climate change, rapid urbanisation and the ageing labour force, the spotlight has turned to the potential of automated timber construction. Industry has long been looking for new ways to build with timber in an economically and environmentally sustainable way. At the same time, satisfying architectural, functional, aesthetical and comfort requirements. However, their solution to this date feels overwhelmingly monotonic -- prefabrication. While I do not doubt that prefabrication will be the solution for a long time ahead, the research community has to look one step ahead.

This thesis is not aimed at solving engineering problems, even though many problems were analysed and solved along the way. Instead, this thesis offers the indulgence to have a daydream to imagine what the future of timber construction could be. I realised the secret to having a good daydream is to be half-awake and half-hallucinating. When I am dreaming, my mind wanders into science fiction for inspiration. From time to time, I sober up to have a reality check. This thesis is a daydream of what could be, and I hope it will inspire the readers to imagine further.

My journey into this PhD was born out of a mix of personal interest in timber construction and mechatronics. However, over the course of this thesis, I realised the profound implications of marrying automation with timber assembly. It was a journey paved with challenges, discoveries, and invaluable lessons. Every step of the research was a learning experience, revealing layers of complexities and opportunities within the field.

I owe a deep gratitude to my advisors, Prof. Mathias Kohler, Prof. Fabio Gramazio, and Prof. Agathe Koller. Their guidance, patience, and insights were instrumental in shaping this work. Not to mention their tolerance and support of my daydreaming approach, it is what made the discovery of \textit{Distributed Robotic Tools} possible. To my peers, friends, and family - your unwavering support and encouragement have been a constant source of strength. I am also grateful to the National Centres of Competence in Research and the National Science Foundation for funding this research.

I hope this thesis provides readers with a fresh perspective on the potential of \textit{Spatial Timber Assembly} and \textit{Distributed Robotic Tools}. I would like to encourage future researchers to approach this thesis like an unfinished daydream and to continue the journey of discovery where I left off. I have documented all my thoughts, experiments and findings in this thesis to the best of my ability with the hope that it will be useful to future researchers. I made a conscious decision to include all the failures and dead-ends I encountered along the way. I believe it is important to document the failures as much as the successes. A thematic reading guide is provided in \noseeref{section:methodology-thematic-reading-guide} to help readers navigate the thesis. May it spark further inquiries, innovations, and discussions. 

Victor Leung

\end{adjustwidth}
