\chapter{Introduction}

The advancement of technology has opened new avenues for innovation in construction.
The use of robots in the assembly of timber structures presents a new frontier in this field.
This interdisciplinary PhD thesis focuses on the integration of industrial robots and distributed robotic tools (DiRT) for the automatic assembly of timber structures.
Specifically, timber frame structures with integral timber joints. 

Despite the potential benefits that automation could bring to the construction industry, the necessary technology to achieve this goal remains unknown. The objective of this thesis is to embark on a journey of discovery and innovation that will contribute to the advancement of knowledge in the field.
By combining interdisciplinary approaches from timber construction, computational design and robotics, this research will provide insights into the technical and operational aspects of the automatic assembly process, as well as its potential benefits and challenges.
Through the development of novel solutions and empirical construction experiments, this PhD thesis aspires to shape the future of construction and contribute to the interdisciplinary field of digital fabrication, between architecture, construction and robotics.

\section{Timber Frame Construction}

Timber construction has been used historically in many cultures with an ample supply of trees.
Timber is a remarkable construction material because of its very high strength-to-weight ratio.
It can withstand tension and compression, making it suitable for different structural roles. Moreover, timber can naturally reach lengths of many metres, which provides tall columns and long beams for the construction of architectural structures at various scales.
Historically, many types of construction methods have been developed. The focus of this thesis is timber frame construction which refers to a range of timber construction methods used as early as the 12th century until the early 20th century (Sobon n Schroeder, 1984).

\subsection{Defining Characteristics}
We can understand timber frame construction as a set of techniques that have evolved over time. Carpenters have experimented with different joint designs, structural arrangements and methods to frame walls, floors and roofs. While these techniques evolved independently in different civilisations, similar material behaviour (e.g. anisotropicity of wood, prone to decay) and similar design goals (e.g. weather protection) resulted in many similarities between them (Zwerger, 2012). While there are still differences between regional practices (caused by regional timber species, weather, local culture and ornamental preferences), the type of timber construction that dominated the pre-industrial era (especially in Europe, North America, and Japan) share similar defining characteristics: (1) Large timber sizes, (2) integral timber joints, and (3) post-and-beam framing. 
These characteristics make timber frame construction ideal for studying automatic assembly. As such, this thesis is not referring to the timber frame construction of a specific location or time period, but rather to the general use of these three defining characteristics of timber frame construction.

\subsubsection{Large Timber Size}
The first defining characteristic of timber frame construction is large timber sizes. Historically, this is the result of using a whole harvested tree trunk as a structural member instead of laboriously cutting them into smaller pieces. It was only after the industrialisation of sawmills that timber could be efficiently cut into smaller dimensions. Around the same time, the depletion of large trees in Europe and America prompted the more contemporary invention of stick framing construction (i.e. balloon framing and platform framing) that uses dimensional lumber (e.g. 2x4 inch “two-by-four” studs). Later as timber glue technology improved, the use of glue-laminated timber (glulam), laminated veneer lumber (LVL) and other engineered timber products began to appear and allowed the manufacturing and use of large timber elements again. In this thesis, the assembly of both solid timber and engineered timber is investigated. The nominal size being used is 100mm x 100mm square profile (see 4.1.3 Scope for Initial Development Round)

\subsubsection{Integral Timber Joints}
The second defining characteristic in timber frame construction is the use of integral timber joints (ITJ), where the jointing geometry is carved into the material. Structural load is transmitted from element to element directly through contact pressure over the mating surfaces of the joint. The design of these joints has been refined empirically throughout history, converging into specialised designs according to which structural elements they are connecting (Jack Sobon, 2014). It is common for a joint to serve multiple purposes and thus withstands different types of loads, such as self-weight and live forces. Different ‘geometrical features’ can therefore be incorporated into the connection to meet different force conditions. For example, a lap joint for a floor joist can be combined with a dovetail feature to provide tension resistance. These joints are typically dry-fitted (footnote: because historical glue technology could not withstand structural forces) and held in place due to interlocking geometry and gravity. Additionally, it is possible to include metal components in integral timber joints. This is more common in contemporary design, where metal screws, dowels and tie rods are used to improve joint capacity or to reduce the need to cut complicated joinery. For the purpose of studying automatic assembly, integral timber joints provide a unique advantage because they can be prefabricated with high accuracy (see 1.1.2.1 Automatic machining of parametric timber joints) and are quick to install. In this thesis, the assembly of dry-fitted joints both with and without fasteners is explored.
As of today, there is no single consensus in the nomenclature of timber joints as different cultures have historically named their joints using different methods (Sato and Nakahara, 1995; Seike, 1977; Sumiyoshi and Matsui, 1991). The first naming convention (commonly used in American and European traditions) is based on the location of the joint and the structural elements it connects to, for example, ‘floor joist joint’ or ‘tying joint below plate’. The second naming convention is based on the joint's geometrical features, for example, ‘dovetail joint’ or ‘scarf lap joint’. Some unique joint designs could also be named according to the use in a famous building. For example, Osaka-jo-otemon-hikae-bashira-tsugite refers to Osaka Castle-Otemon Gate's pillar splice. For the purpose of studying assembly problems, this thesis proposes the use of a different convention that is based on the assembly direction related to the joint (see Section 4.2.1 Lap Joint Classification by Assembly Direction).
The photo below shows timber frame components being assembled on the ground. Notice the use of large timber sizes and integral timber joints. 

(Credits: CC AS 4.0 Licence photo by Georg Hefter - Traditionelles Fachwerk und Timberframe im Vergleich auf georghefter.de)

\subsubsection{Post and Beam Structure}
The third defining characteristic of timber frame construction is post-and-beam structural framing (Jack Sobon, 2014; Sobon and Schroeder, 1984). The use of large timber sizes in timber frame structures allows them to be strong enough without the use of continuous load-bearing walls. The main benefit is the flexible arrangement of floors and walls to suit different architectural programmes. 
On the other hand, integral timber joints are generally weak in rotational stiffness, causing the joints to behave kinematically. Therefore, diagonal bracing is often introduced to fully or partially triangulate the structure to improve structural stiffness. Being limited by the joint design, these bracings function primarily in compression. It is, therefore, common to design the bracings in complementary pairs to resist dynamic wind loads coming from opposing directions. While some architectural designs may find the intrusion of diagonal bracings obtrusive to usable space, these diagonal bracings are highlighted on the facade as ornamentation in half timber designs \cite{@gernerFachwerkEntwicklungGefuege1979} (Gerner, 1979; Skinner, 2007). In this thesis, the demonstration structures are designed with similar structural principles, specifically the use of structural triangulation in the design offered unique stability advantages during the robotic construction process (see 7.1.1 Deformation-Awareness and Error Correction by Triangulation and 7.5.2.6 Global Correction Approach).
The image below shows an example of a half-timber style timber frame house in Soest, Germany. Notice the exposed diagonal elements on the facade.
(Credits: Licence-free photo, retrieved from https://www.hippopx.com/en/germany-soest-architecture-timber-framed-half-timbered-house-square-city-402214 , n.d.)
